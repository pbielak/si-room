\pagebreak
\section{Podsumowanie}
\begin{itemize}
  \item{Dla problemu optymalizacji funkcji Rastrigina bez ograniczeń, najszybciej optymalne rozwiązania znajduje algorytm PSO (około 20 iteracji).
        Algorytm FFA znajdował to rozwiązanie po 80 iteracjach, natomiast BAT po 110 iteracjach. Dodatkowo można zauważyć, że w PSO wszystkie osobniki
        ostatecznie zbiegają do rozwiązania optymalnego. W przypadku BAT osiągają wartość bliską optimum, ale od pewnego momentu średnia ocena rozwiązań
        zaczyna się pogarszać. Dla FFA część świetlików utyka w minimach lokalnych, co skutkuje gorszym średnim rozwiązaniem.}
  \item{W przypadku wprowadzenia ograniczeń do funkcji Rastrigina, PSO zachowuje się dość podobnie (nieznaczny wpływ). Można zauważyć znaczny wpływ na
        charakterystykę zbiegania algorytmu FFA -- średnie rozwiązanie jest lepsze, ale nadal występuje problem utykania w minimach lokalnych; dodatkowo
        wcześniej jest znajdywane rozwiązanie optymalne -- po około 30 iteracjach. Dla algorytmu BAT można zauważyć wydłużenie czasu (liczby iteracji),
        po których zostaje znalezione optimum.}
  \item{Problem projektowania pokoju jest znacznie bardziej skomplikowany niż w przypadku funkcji Rastrigina. Dla algorytmów PSO oraz BAT można zauważyć,
        że zarówno wartość oceny najlepszego rozwiązania i średniego rozwiązania wzrastają wykładniczo w zależności od liczby iteracji. Dodatkowo wszystkie
        osobniki w algorytmie BAT zbiegają w końcu do najlepszego znalezionego rozwiązania. W przypadku algorytmu FFA widać, że najlepsze znalezione 
        rozwiązanie jest gorsze niż w pozostałych algorytmach i wartość oceny średniego rozwiązania jest dość niska.}  
  \item{Czas wykonywania pojedynczej iteracji był dość krótki (< 1s) w większości przypadków, jednak w trakcie badań problem pojawił się w algorytmie FFA
        dla problemu projektowania pokoju. Średni czas pojedynczej iteracji był na poziomie 3-4 sekund. Można to tłumaczyć stosunkowo złożoną procedurą oceny
        pokoju (funkcja oceny) w połączeniu ze złożonością algorytmu FFA -- $O(n^2)$ dla wewnętrznej pętli zależnej od liczby świetlików.}
\end{itemize}
